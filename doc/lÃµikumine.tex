\documentclass[12pt,a4paper]{report}
\usepackage[utf8]{inputenc}
\usepackage[estonian]{babel}
\begin{document}
  \section{Kiire lõikumine kolmurgaga}
  Olgu \(A\) kolmnurga tipp ja punkt \(P_0\) kiire alguspunkt.
   
  Otsitav punkt \(P\) peab olema kolmnurgaga koplanaarne ja kiirega
  kollineaarne. Seega:
  \[\left\{\begin{array}{l}
      (P-A)\cdot\vec n = 0\\
      P=P_0+k\vec s
    \end{array}\right.\]
  Asendan kiire võrrandi tasandi võrrandisse.
  \[(P_0+k\vec s-A)\cdot\vec n = 0\]
  Avan sulud. \(P_0-A\) on vektor.
  \[k\vec s\cdot\vec n+(P_0-A)\cdot\vec n = 0\]
  \[k\vec s\cdot\vec n = (A-P_0)\cdot\vec n\]
  Avaldan \(k\).
  \[k=\frac{(A-P_0)\cdot\vec n}{\vec s\cdot\vec n}\]
  Kui \(k = \frac n0\), siis on tasand ja kiir paralleelsed ja lõikumist ei
  toimu. Kui \(k = \frac00\), siis asub kiir tasandil ja lõikumine toimub igas
  kiire punktis. Kui \(k < 0\), on tasand kiire taga ja lõikumist ei toimu.
  
  Punkt \(P\):
  \[P=P_1+\frac{(A-P_0)\cdot\vec n}{\vec s\cdot\vec n}\cdot\vec s\]
  
\end{document}
