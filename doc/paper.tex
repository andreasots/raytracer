\documentclass[a4paper,12pt]{report}
\usepackage{amsmath}
\usepackage[estonian]{babel}
\usepackage[bf]{caption}
\usepackage[top=2.54cm, bottom=2.54cm, left=3cm, right=2cm]{geometry}
\usepackage[utf8]{inputenc}
\usepackage{sectsty}
\usepackage[compact]{titlesec}
\usepackage{wrapfig}
\chapterfont{\fontsize{16pt}{19pt}\sffamily}
\linespread{1.3}
\renewcommand{\thechapter}{\arabic{chapter}.}
\renewcommand{\thesection}{\thechapter\arabic{section}.}
\renewcommand{\thesubsection}{\thesection\arabic{subsection}.}
\renewcommand{\thesubsubsection}{\thesubsection\arabic{subsubsection}.}
\renewcommand{\theequation}{\thechapter\arabic{equation}}
\renewcommand{\thefigure}{\thechapter\arabic{figure}}

% et nad kaklema ei läheks
\usepackage[colorlinks=true]{hyperref}
\usepackage[tocbib]{apacite}

\pagestyle{empty}
\begin{document}
\begin{center}
Tallinna Reaalkool

\vfill

Kiirtejälitus\\
Uurimistöö

\vfill

\end{center}

\begin{raggedleft}

Andreas Ots

11a

Juhendaja: õp Inga Petuhhov

\end{raggedleft}

\vfill

\begin{center}

Tallinn 2011

\end{center}
\clearpage

\addtocontents{toc}{\protect\thispagestyle{empty}}
\tableofcontents
\listoffigures
\listoftables

\chapter*{Sissejuhatus}
\addcontentsline{toc}{chapter}{Sissejuhatus}
\pagestyle{plain}
\thispagestyle{empty}
Esimene lehekülg

\clearpage

Teine lehekülg

\chapter{Kiirtejälitus}
\section{Visualiseerimisvõrrand}
Aastal 1986 pakkus James Kajiya välja võrrandi erinevate
visualiseerimisalgoritmide ül\-dis\-ta\-mi\-seks
(võrrand~\ref{eq:rendering-kajiya86}). \cite{kajiya86}

\begin{equation} \label{eq:rendering-kajiya86}
I(X, X') = g(X, X')\left[\epsilon(X, X')+\int_S\rho(X, X', X'')
I(X', X'')dX''\right]
\end{equation}

\begin{tabular}{l p{0.75\textwidth}}
\(I(X, X')\) & Valguse heledus, mis jõuab punktist \(X'\) punkti \(X\)\\
\(g(X, X')\) & Geomeetriline tegur. Näitab, kui palju punktist \(X'\)
	väljuvat valgust jõuab punkti \(X\) \\
\(\epsilon(X, X')\) & Valguse heledus, mis kiirgub punktist \(X'\)
	punkti \(X\) \\
\(\rho(X, X', X'')\) & Peegeldustegur. Näitab, kui palju punktist
	\(X''\) väljuvat valgust jõuab läbi punkti \(X'\) punkti \(X\) \\
\(S\) & Esemete hulk, millest koosneb visualiseeritav stseen \\
 & \\
\end{tabular}


Probleem selle valemiga on see, et seob lineaarselt pinna heleduse
esemete arvuga. Üks võimalus on võrrand \ref{eq:rendering-kajiya86}
kirjutada vektorkujul (võrrand \ref{eq:rendering-vec}).

\begin{equation} \label{eq:rendering-vec}
I(X, \omega) = \epsilon(X, \omega) + \int_\Omega\rho(X, \omega', \omega)
I(X, \omega')d\omega'
\end{equation}
\[\Omega = \{\omega ~|~ \omega\cdot \vec n \leq 0,~ \omega^2 = 1\}\]
\[\omega \cdot \vec n \leq 0\] 

\begin{tabular}{l p{0.75\textwidth}}
\(I(X, \omega)\) & Valguse heledus, mis väljub punktis \(X\) suunast
	\(-\omega\) \\
\(\epsilon(X, \omega)\) & Valguse heledus, mis kiirgub punktist \(X\) suunas
	\(-\omega\)\\
\(\rho(X, \omega', \omega)\) & Peegeldustegur. Näitab, kui palju suunast
	\(\omega'\) peegelduvat valgust jõuab peegeldub punktis \(X\) suunas
	\(-\omega\) \\
\(\vec n\) & Pinna normaal punktis \(X\) \\
 & \\
\end{tabular}



\chapter*{Kokkuvõte}
\addcontentsline{toc}{chapter}{Kokkuvõte}

% Kasutatud materjalid
\renewcommand\bibname{Kasutatud materjalid}
\bibliographystyle{apacite}
\bibliography{paper}
\end{document}
