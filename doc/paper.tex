\documentclass[a4paper,12pt]{report}
\usepackage{amsmath}
\usepackage[estonian]{babel}
\usepackage[bf]{caption}
\usepackage[top=2.54cm, bottom=2.54cm, left=3cm, right=2cm]{geometry}
\usepackage[utf8]{inputenc}
\usepackage{sectsty}
\usepackage[compact]{titlesec}
\usepackage{wrapfig}
\usepackage{esvect}
\renewcommand{\vec}[1]{\vv{#1}}
\usepackage{listings}
\lstset{basicstyle=\footnotesize, language=C++, numbers=left, numberbychapter=false}
\chapterfont{\fontsize{16pt}{19pt}\sffamily}
\linespread{1.3}
\renewcommand{\thechapter}{\arabic{chapter}.}
\renewcommand{\thesection}{\thechapter\arabic{section}.}
\renewcommand{\thesubsection}{\thesection\arabic{subsection}.}
\renewcommand{\thesubsubsection}{\thesubsection\arabic{subsubsection}.}
\renewcommand{\theequation}{\thechapter\arabic{equation}}
\renewcommand{\thefigure}{\thechapter\arabic{figure}}
\renewcommand{\lstlistlistingname}{Programmid}
\renewcommand{\lstlistingname}{Programm}

% et nad kaklema ei läheks
\usepackage[colorlinks=true]{hyperref}
\usepackage[tocbib]{apacite}

\pagestyle{empty}
\begin{document}
\begin{center}
Tallinna Reaalkool

\vfill

Kiirtejälitus\\
Uurimistöö

\vfill

\end{center}

\begin{raggedleft}

Andreas Ots

11a

Juhendaja: õp Inga Petuhhov

\end{raggedleft}

\vfill

\begin{center}

Tallinn 2011

\end{center}
\clearpage

\addtocontents{toc}{\protect\thispagestyle{empty}}
\tableofcontents
\listoffigures
\listoftables
\lstlistoflistings

\chapter*{Sissejuhatus}
\addcontentsline{toc}{chapter}{Sissejuhatus}
\pagestyle{plain}
\thispagestyle{empty}
Esimene lehekülg

\clearpage

Teine lehekülg

\chapter{Kiirtejälitus}
\section{Kiirtejälituse mõiste}
Järgnev lõik pärineb Eesti Standardikeskuse standardist EVS-ISO/IEC~2382-13:1998:
\begin{quote}
Kiirte jälitus --- Vaatleja silmast stseeni objektideni kulgevate
kujuteldavate valguskiirte jälitusel põhinev meetod stseeni nende osade
määramiseks, mis tuleb kuvada saadaval pildil.\cite{ISO:2382-13}
\end{quote}

Kiirtejälitus pakub praeguseni teadaolevatest ilmestusmeetoditest kõige
realistlikumat tulemust, sest kasutab ilmestusel valguse füüsikalist
mudelit.


\section{Valguse füüsikaline mudel}
Kiirtejälitus kasutab radiomeetriat valguse mudelina, sest piksli heledus
ekraanil on otseselt seotud mõõdetud kirkusega.

Läbi stseeni peegeldunud valguse kirkuse määramiseks on vaja arvutada
läbi pilditasandi paistvate pindade kirkust. Selleks on muuhulgas vaja
ka teada, kuidas muutub peegelduva valguse suund valgusallikast vaatlejani
jõudes. Funktsioon \(f_r(\omega', \omega)\) näitab, kui palju suunast
\(-\omega'\) tulnud valgusest peegeldub suunda \(\omega\). Radiomeetrilistes
suurustes tähendab see väljuva kirkuse suhe sissetulevasse kiiritustihedusse
(võrrand \ref{eq:BRDF}).

\begin{equation} \label{eq:BRDF}
f_r(\omega', \omega) = \frac{dL_r(\omega)}{dE_r(\omega')}
\end{equation}

Funktsioon \(f_r(\omega', \omega)\) näitab materjali omadust peegeldada
valgust. Kõikidel füüsikaliselt korrektsetel materjalidel on kolm
lisaomadust:

\begin{enumerate}
\item Positiivsus: \(f_r(\omega', \omega) \geq 0 \)
\item Ümberpööratavus: \(f_r(\omega', \omega) = f_r(\omega, \omega')\)
\item Energia jäävus: \(\int_\Omega f_r(\omega', \omega) d\omega \leq 1\) iga \(\omega'\) korral
\end{enumerate}

Funktsiooni \(f_r(\omega', \omega)\) abil on võimalik leida punktist \(x\)
väljuv kirkus \(L_o\) (võrrand \ref{eq:reflectance}). \(x\) on mõõdetav
punkt, \(\omega\) on vektor mõõdetavast punktist vaatleja poole,
\(L_i\) on suunast \(\omega'\) tulev kirkus.

\begin{equation} \label{eq:reflectance}
L_o(x, \omega) = \int_\Omega f_r(x, \omega', \omega) L_i(x, \omega') d\omega'
\end{equation}

Funktsiooni \(f_r(\omega', \omega)\) sõltuvus mõõdetavast punktist on
seletatav sellega, et see funktsioon määrab ühe kindla pinna
peegelduvuse, mitte kõikide stseenis olevate pindade peegelduvust.

Võrrand \ref{eq:reflectance} ei ole piisav, kui pind ka kiirgab valgust.
Kiiratava valguse kirkuse (\(L_e\)) võib lihtsalt juurde liita peegeldunud valguse
kirkusele (võrrand \ref{eq:rendering}).

\begin{equation} \label{eq:rendering}
L_o(x, \omega) = L_e(x, \omega) + \int_\Omega f_r(x, \omega', \omega) L_i(x, \omega') d\omega'
\end{equation}

Ainus seni käsitlemata tegur võrrandis \ref{eq:rendering} on \(L_i\),
mille saab avaldada \(L_o\) kaudu (võrrand \ref{eq:incident}). 

\begin{equation} \label{eq:incident}
L_i(x, \omega) = L_o(x + t\omega, -\omega)
\end{equation}

Kui asendada võrrand \ref{eq:incident} võrrandisse \ref{eq:rendering},
on tulemuseks mitmekordne integraal, mida on isegi arvutil ebamugav
analüütiliselt lahendada.

\section{Monte Carlo meetod}
Monte Carlo meetod on stohhastiline meetod integraali väärtuse arvutamiseks.
Igast integreerimislõigust võetakse juhuslik suurus ja leitakse integraali
ligikaudne väärtus. Kõikide leitud väärtuste aritmeetiline keskmine
läheneb integraali väärtusele kui ligikaudsete tulemuste arv läheneb
lõpmatusele. Monte Carlo meetodiga on võimalik suhteliselt lihtsalt
ligikaudselt määrata ka keerulisi integraale nagu näiteks võrrand \ref{eq:rendering}.

Üks puudus eelmises lõigus kirjeldatud meetodil siiski on: kui juhusliku
protsessiga jõutakse kohale, kus funktsiooni väärtus on väga suur, siis
Monte Carlo meetod ülehindab integraali väärtust ja pildile tuleb teistega
võrreldes väga hele piksel (nn ``jaaniussike"). Selle ennetamiseks tuleks
integreerimislõigust valimisel eelistada väärtusi, mille korral
integreeritava väärtus on suurem. Sellisel juhul tuleb integreeritavat
läbi jagada juhusliku väärtuse tõenäosusfunktsiooniga (võrrand \ref{eq:mc-is1}).

\begin{equation} \label{eq:mc-is1}
L_o(x, \omega) = L_e(x, \omega) + \int_\Omega \frac{f_r(x, \omega', \omega)}{P(\omega')} L_i(x, \omega') d\omega'
\end{equation}

Kui valida võrrandis \ref{eq:mc-is1} \(\omega'\) nii, et selle
tõenäosusfunktsioon on võrdne peegeldusfunktiooniga, siis on nende
jagatis üks ja võrrand lihtsustub (võrrand \ref{eq:mc-is2}).

\begin{equation} \label{eq:mc-is2}
L_o(x, \omega) = L_e(x, \omega) + \int_\Omega L_i(x, \omega') d\omega'
\end{equation}

\section{Lähima pinna leidmine}
Olgu kolmemõõtmelises ruumis kaks punkti \(A\) ja \(B\). Kõik punktid
lõigul \(AB\) on võimalik leida, liites punktile \(A\) vektori \(\vec{AB}\)-ga
samasihilise, kuid lühema vektori (võrrand \ref{eq:ray-1}).

\begin{equation} \label{eq:ray-1}
P = A + t\vec{AB}, \quad 0 \leq t \leq 1
\end{equation}

Kiire definitsioon ütleb, et kiir on lõik, mis on pikendatud lõpmatult
üle ühe otspunkti. Pikendades üle punkti \(B\) peab olema \(t\)
mittenegatiivne, mida on lihtsam kontrollida, kui seda, kas \(t\) on
väiksem ühest.

Uurimistöös on kasutusel selline tähistus kui pole märgitud teisiti:
\(P\) on kiire alguspunkt, \(\vec s\) on ühikvektor kiire suunas ja
\(P'\) on punkt kiirel, mille kaugus kiire alguspunktist on \(t\)
(võrrand \ref{eq:ray}).

\begin{equation} \label{eq:ray}
P' = P + t\vec s, \quad t \geq 0
\end{equation}

Võrrandis \ref{eq:incident} on \(L_o\) esimene argument sarnane kiire
võrrandiga, nimelt on seal \(t\) vähim positiivne\footnote{Kui \(t=0\),
saadaks jälle punkt \(x\)} arv, mille korral \(x + t\omega\) on lähim
punkt mõnel pinnal. Selle punkti saab leida, kui leida kõikide pindade
lõikepunktid selle kiirega ja võtta neist vähim positiivne.

\subsection{Kiire lõikumine tasandiga}
Kõik vektorid tasandil on risti tasandi normaalvektoriga (võrrand
\ref{eq:plane1}). \(A\) on punkt tasandil ja \(P'\) on otsitav punkt.

\begin{equation} \label{eq:plane1}
(A - P') \cdot \vec n = 0
\end{equation}

Kiire ja tasandi lõikepunkti leidmiseks on vaja lahendada võrrandisüsteem
\ref{eq:plane2}. Parameetri \(t\) kohta kehtivad samad piirangud, mis
võrrandis \ref{eq:ray}.

\begin{equation} \label{eq:plane2}
\left\{
\begin{array}{l}
(A - P') \cdot \vec n = 0 \\
P' = P + t\vec s
\end{array}
\right.
\end{equation}

\[(A - P + t\vec s) \cdot \vec n = 0\]

\[(A - P) \cdot \vec n + t\vec s \cdot \vec n = 0\]

\[t\vec s \cdot \vec n = (P-A) \cdot \vec n\]

\begin{equation} \label{eq:plane3}
t = \frac{(P-A) \cdot \vec n}{\vec s \cdot \vec n}
\end{equation}

Kui võrrandis \ref{eq:plane3} on lugeja null, algab kiir tasandilt, kui
nimetaja on null, siis kiir on paralleelne tasandiga, kui aga mõlemad on
nullid, asub kiir tasandil.

\subsection{Kiire lõikumine kolmnurgaga}
Olgu kolmnurk \(ABC\). Olgu punkt \(A\) koordinaatsüsteemi alguspunkt ja
vektorid \(\vec{AB}\) ja \(\vec{AC}\) vektorruumi baasiks. Nüüd on
võimalik kirjutada iga punkt selle kolmnurga tasandil nende vektorite
summana (võrrand \ref{eq:tri-1}).

\begin{equation} \label{eq:tri-1}
P' = A + u\vec{AB} + v\vec{AC}
\end{equation}

Kui \(u\) või \(v\) on väiksem kui null või suurem kui üks, asub punkt
kolmnurgast väljas. Kui punkt asub sirge \(BC\) kohal, on punkt
kolmnurgast väljas. Sirge \(BC\) võrrand on \(x + y = 1\), seega peab
punkti koordinaatide summa olema väiksem või võrdne ühega.

\subsubsection{Möller-Trumbore lähenemine}
Kiire lõikumise kolmnurgaga saab leida võrrandisüsteemist \ref{eq:tri-mt-1}.

\begin{equation} \label{eq:tri-mt-1}
\left\{
\begin{array}{l}
P' = P + t\vec s \\
P' = A + u\vec{AB} + v\vec{AC} \\
0 \leq u \leq 1 \\
0 \leq v \leq 1 \\
u + v \leq 1 \\
t > 0 
\end{array}
\right.
\end{equation}

\[P + t\vec s = A + u\vec{AB} + v\vec{AC}\]

\[u\vec{AB} + v\vec{AC} -t\vec s = (P - A)\]

\begin{equation} \label{eq:tri-mt-2}
u\vec{AB} + v\vec{AC} + t(-\vec s) = (P - A)
\end{equation}

Võrrand \ref{eq:tri-mt-2} on kolme tundmatuga lineaarvõrrandisüsteem,
mida on võimalik lahendada kolmerealise determinandiga.

\subsubsection{Waldi lähenemine}
Leida punkt \(P'\) kiire ja tasandi lõikumise võrrandist (võrrand
\ref{eq:plane3}), projitseerida kolmnurk ja leitud punkt mõnele koordinaatteljega
risti olevale tasandile ja lahendada võrrand \ref{eq:tri-1}.

\subsection{Kiire lõikumine keraga}
\begin{equation} \label{eq:sphere1}
\left\{
\begin{array}{l}
(A - P')^2 = r^2 \\
P' = P + t\vec s
\end{array}
\right.
\end{equation}

\subsection{Kiire lõikumine silindriga}
Silinder otspunktidega \(A\) ja \(B\) ning raadiusega \(r\). Punkti \(P'\) kaugus sirgest \(AB\)
peab olema \(r\). Punkt \(P'\) ei tohi olla punktist \(A\) kaugemal kui \(B\) vektori \(\vec{AB}\)
sihis ja \(P'\) ei tohi olla punktist \(B\) kaugemal kui \(A\) vektori \(\vec{AB}\) sihis (võrrand \ref{eq:cyl-1}).

\begin{equation} \label{eq:cyl-1}
\left\{
\begin{array}{l}
\frac{\left(\vec{AB}\times\vec{AP'}\right)^2}{\vec{AB}^2} = r^2 \\
0\leq\vec{AB}\cdot\vec{AP'}\leq|\vec{AB}| 
\end{array}
\right.
\end{equation}

\chapter*{Kokkuvõte}
\addcontentsline{toc}{chapter}{Kokkuvõte}

% Kasutatud materjalid
\renewcommand\bibname{Kasutatud materjalid}
\bibliographystyle{apacite}
\bibliography{paper}

\appendix
\chapter{Radiomeetria mõisted}
\begin{description}
\item[Kiirgusvoog] (ing. k. \textit{radiant flux}) on energiahulk, mille kiirgus kannab ajaühikus läbi pinna.
\item[Kiiritustihedus] (ing. k. \textit{irradiance}) iseloomustab ruumi- ja mittepunktvalgusallikate kiiratud valgust.
\item[Kirkus] (ing. k. \textit{radiance}) nimetatakse pinnale langevat kiirgusvoogu.
\end{description}
\end{document}
