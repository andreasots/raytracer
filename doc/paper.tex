\documentclass[a4paper,12pt]{report}
\usepackage{amsmath}
\usepackage[estonian]{babel}
\usepackage[bf]{caption}
\usepackage[top=2.54cm, bottom=2.54cm, left=3cm, right=2cm]{geometry}
\usepackage[utf8]{inputenc}
\usepackage{sectsty}
\usepackage[compact]{titlesec}
\usepackage{wrapfig}
\chapterfont{\fontsize{16pt}{19pt}\sffamily}
\linespread{1.3}
\renewcommand{\thechapter}{\arabic{chapter}.}
\renewcommand{\thesection}{\thechapter\arabic{section}.}
\renewcommand{\thesubsection}{\thesection\arabic{subsection}.}
\renewcommand{\thesubsubsection}{\thesubsection\arabic{subsubsection}.}
\renewcommand{\theequation}{\thechapter\arabic{equation}}
\renewcommand{\thefigure}{\thechapter\arabic{figure}}

% et nad kaklema ei läheks
\usepackage[colorlinks=true]{hyperref}
\usepackage[tocbib]{apacite}

\pagestyle{empty}
\begin{document}
% apacite tõlked
\renewcommand{\BUPhD}{Avaldamata doktoritöö}
\renewcommand{\BAvailFrom}{Loetud:\ }
\begin{center}
Tallinna Reaalkool

\vfill

Kiirtejälitus\\
Uurimistöö

\vfill

\end{center}

\begin{raggedleft}

Andreas Ots

11a

Juhendaja: õp Inga Petuhhov

\end{raggedleft}

\vfill

\begin{center}

Tallinn 2011

\end{center}
\clearpage

\addtocontents{toc}{\protect\thispagestyle{empty}}
\tableofcontents
\listoffigures
\listoftables

\chapter*{Sissejuhatus}
\addcontentsline{toc}{chapter}{Sissejuhatus}
\pagestyle{plain}
\thispagestyle{empty}
Esimene lehekülg

\clearpage

Teine lehekülg

\chapter{Kiirtejälitus}
\section{Kolmnurga ja kiire lõikumine}

\subsection{Tasand}
\[
  \left\{
  \begin{array}{l}
    P' = P + t\vec s \\
    (P' - A)\cdot\vec n = 0
  \end{array}
  \right.
\]

\[(P + t\vec s - A)\cdot\vec n = 0\]

\[(P - A)\cdot\vec n + t\vec s\cdot\vec n = 0\]

\[t\vec s\cdot\vec n = (A - P)\cdot\vec n\]

\[t = \frac{(A - P)\cdot\vec n}{\vec s\cdot\vec n}\]

Kui \(t = \pm\inf\), kiir ei lõiku tasandiga. Kui \(t = \frac00\)\footnote{IEEE~
754 standardi järgi \(\frac00=\)\texttt{NaN}.}

\subsection{Barütsentrilised koordinaadid}
\begin{wrapfigure}{r}{0.5\textwidth}
  \begin{center}
    \setlength{\unitlength}{5cm}
    \begin{picture}(1.25,0.75)
      \put(0.125,0.125){\line(1,0){1}}
      \put(1.125,0.125){\line(-5,4){0.715}}
      \put(0.125,0.125){\line(1,2){0.285}}
      \put(0,0){\(A(1,0,0)\)}
      \put(1,0){\(B(0,1,0)\)}
      \put(0.35,0.75){\(C(0,0,1)\)}
      \put(0.5,0.35){\circle*{.03}}
      \put(0.3, 0.4){\(P(\lambda_1, \lambda_2, \lambda_3)\)}
    \end{picture}
    \[\forall i \in \{1,2,3\} \quad 0 \geq \lambda_i \geq 1\]
  \end{center}
  \caption{Barütsentrilised koordinaadid kolmnurgas}
  \label{fig:bary}
\end{wrapfigure}
Barütsentrilised koordinaadid on ar\-vu\-kol\-mik \((\lambda_1, \lambda_2,\lambda_3\),
mis tähistavad raskuseid etalonkolmnurga \(\triangle ABC\) tippudes. Need
raskused omakorda määravad punkti \(P\), mis asub nende raskuste raskuskeskmes.
Ba\-rü\-tsent\-ri\-li\-sed koordinaadid avastas August Ferdinand Möbius aastal 1827.
Homogeensete barütsentriliste koordinaatide korral kehtib valem \ref{eq:bary}.
\cite{mw:bary} Joonis \ref{fig:bary} näitab barütsentrilisi koordinaate
kolmurgas.

\begin{equation} \label{eq:bary}
  \lambda_1 + \lambda_2 + \lambda_3 = 1
\end{equation}

Olgu \(A(x_1, y_1, z_1)\), \(B(x_2, y_2, z_2)\) ja \(C(x_3, y_3, z_3)\) mingi
kolmnurga tipud ja punkt \(P(x, y, z)\) selle kolmurgaga samas tasandis asuv
punkt.

Punkti \(P\) saab määrata kolmnurga \(ABC\) barütsentriliste koordinaatidega
\((\lambda_1, \lambda_2, \lambda_3)\). Barütsentrilised koordinaadid saab
teisendada eukleidilise ruumi koordinaatideks valemiga \ref{eq:bary-1} või
komponetide kaupa võrrandisüsteem \ref{eq:bary-1-exp}.

\begin{minipage}{0.40\textwidth}
  \begin{equation} \label{eq:bary-1}
    P = \lambda_1A+\lambda_2B+\lambda_3C
  \end{equation}
\end{minipage}
\begin{minipage}{0.55\textwidth}
  \begin{equation} \label{eq:bary-1-exp}
    \left\{
    \begin{array}{l}
      x = \lambda_1x_1+\lambda_2x_2+\lambda_3x_3 \\
      y = \lambda_1y_1+\lambda_2y_2+\lambda_3y_3 \\
      z = \lambda_1z_1+\lambda_2z_2+\lambda_3z_3
    \end{array}
    \right.
  \end{equation}
\end{minipage}

Võrrandisüsteem \ref{eq:bary-2} on süsteemi \ref{eq:bary-1} lahend.

\begin{center}
\begin{minipage}{0.23\textwidth}
  \[
    D =
    \begin{vmatrix}
      x_1 & x_2 & x_3 \\
      y_1 & y_2 & y_3 \\
      z_1 & z_2 & z_3 \\
    \end{vmatrix}
  \]
\end{minipage}
\begin{minipage}{0.23\textwidth}
  \[
    D_{\lambda_1} =
    \begin{vmatrix}
      x & x_2 & x_3 \\
      y & y_2 & y_3 \\
      z & z_2 & z_3 \\
    \end{vmatrix}
  \]
\end{minipage}
\begin{minipage}{0.23\textwidth}
  \[
    D_{\lambda_2} = 
    \begin{vmatrix}
      x_1 & x & x_3 \\
      y_1 & y & y_3 \\
      z_1 & z & z_3 \\
    \end{vmatrix}
  \]
\end{minipage}
\begin{minipage}{0.23\textwidth}
  \[
    D_{\lambda_3} = 
    \begin{vmatrix}
      x_1 & x_2 & x \\
      y_1 & y_2 & y \\
      z_1 & z_2 & z \\
    \end{vmatrix}
  \]
\end{minipage}
\end{center}
  
\begin{equation} \label{eq:bary-2}  
  \left\{
  \begin{array}{l}
    \lambda_1 = \frac{D_{\lambda_1}}{D} \\
    \lambda_2 = \frac{D_{\lambda_2}}{D} \\
    \lambda_3 = \frac{D_{\lambda_3}}{D}
  \end{array}
  \right.
\end{equation}

Kuid sellel meetodil üks viga, kolmerealise determinandi arvutamiseks on vaja
teha 12 suhteliselt kallist korrutamistehet. Kaherealise determinandi
arvutamiseks oleks vaja ainult kaks korrutamistehet. Võrrandisüsteemi
\ref{eq:bary-1-exp} esimesed kaks rida kehtivad ka kahemõõtmelise ruumi puhul.
Võrrandisüsteem \ref{eq:bary-1-2D} on teisendatud seose \ref{eq:bary} järgi ja
lihtsustatud. Võrrandisüsteem \ref{eq:bary-3} on süsteemi \ref{eq:bary-1-2D}
lahend.

\begin{equation} \label{eq:bary-1-2D}
  \left\{
  \begin{array}{l}
    x-x_3 = \lambda_1(x_1-x_3)+\lambda_2(x_2-y_3) \\
    y-y_3 = \lambda_1(y_1-x_3)+\lambda_2(y_2-y_3) \\
  \end{array}
  \right.
\end{equation}

\begin{center}
\begin{minipage}{0.3\textwidth}
  \[
    D =
    \begin{vmatrix}
      x_1 - x_3 & x_2 - x_3 \\
      y_1 - y_3 & y_2 - y_3 \\
    \end{vmatrix}
  \]
\end{minipage}
\begin{minipage}{0.3\textwidth}
  \[
    D_{\lambda_1} =
    \begin{vmatrix}
      x - x_3 & x_2 - x_3 \\
      y - y_3 & y_2 - y_3 \\
    \end{vmatrix}
  \]
\end{minipage}
\begin{minipage}{0.3\textwidth}
  \[
    D_{\lambda_2} = 
    \begin{vmatrix}
      x_1 - x_3 & x - x_3 \\
      y_1 - y_3 & y - y_3 \\
    \end{vmatrix}
  \]
\end{minipage}
\end{center}
  
\begin{equation} \label{eq:bary-3}  
  \left\{
  \begin{array}{l}
    \lambda_1 = \frac{D_{\lambda_1}}{D} \\
    \lambda_2 = \frac{D_{\lambda_2}}{D} \\
    \lambda_3 = 1-\lambda_1-\lambda_2
  \end{array}
  \right.
\end{equation}

Selle asemel, et kolmnurk pöörata vajalikku asendisse, mis on kulukas, sest
vajab kolme 4x4 maatriksi korrutamist 1x4 maatriksiga, pakub Wald kolmnurga
projitseerimist ühele kolmest koordinaattasandist. Suurema täpsuse saavutamiseks
tuleks tema sõnul valida tasand, kus projektsiooni pindala on suurim. \cite[lk.~91]{wald::PhD}

\chapter*{Kokkuvõte}
\addcontentsline{toc}{chapter}{Kokkuvõte}

% Kasutatud materjalid
\renewcommand\bibname{Kasutatud materjalid}
\bibliographystyle{apacite}
\bibliography{paper}
\end{document}
