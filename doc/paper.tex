\documentclass[a4paper,12pt]{report}
\usepackage{amsmath}
\usepackage[estonian]{babel}
\usepackage[bf]{caption}
\usepackage[top=2.54cm, bottom=2.54cm, left=3cm, right=2cm]{geometry}
\usepackage[utf8]{inputenc}
\usepackage{sectsty}
\usepackage[compact]{titlesec}
\usepackage{wrapfig}
\usepackage{esvect}
\renewcommand{\vec}[1]{\vv{#1}}
\usepackage{listings}
\lstset{basicstyle=\footnotesize, language=C++, numbers=left, numberbychapter=false}
\chapterfont{\fontsize{16pt}{19pt}\sffamily}
\linespread{1.3}
\renewcommand{\thechapter}{\arabic{chapter}.}
\renewcommand{\thesection}{\thechapter\arabic{section}.}
\renewcommand{\thesubsection}{\thesection\arabic{subsection}.}
\renewcommand{\thesubsubsection}{\thesubsection\arabic{subsubsection}.}
\renewcommand{\theequation}{\thechapter\arabic{equation}}
\renewcommand{\thefigure}{\thechapter\arabic{figure}}
\renewcommand{\lstlistlistingname}{Programmid}
\renewcommand{\lstlistingname}{Programm}

% et nad kaklema ei läheks
\usepackage[colorlinks=true]{hyperref}
\usepackage[tocbib]{apacite}

\pagestyle{empty}
\begin{document}
\begin{center}
Tallinna Reaalkool

\vfill

Kiirtejälitus\\
Uurimistöö

\vfill

\end{center}

\begin{raggedleft}

Andreas Ots

11a

Juhendaja: õp Inga Petuhhov

\end{raggedleft}

\vfill

\begin{center}

Tallinn 2011

\end{center}
\clearpage

\addtocontents{toc}{\protect\thispagestyle{empty}}
\tableofcontents
\listoffigures
\listoftables
\lstlistoflistings

\chapter*{Sissejuhatus}
\addcontentsline{toc}{chapter}{Sissejuhatus}
\pagestyle{plain}
\thispagestyle{empty}
Esimene lehekülg

\clearpage

Teine lehekülg

\chapter{Kiirtejälitus}
\section{Kiirtejälituse mõiste}
Järgnev lõik pärineb Eesti Standardikeskuse standardist EVS-ISO/IEC~2382-13:1998:
\begin{quote}
Kiirte jälitus --- Vaatleja silmast stseeni objektideni kulgevate
kujuteldavate valguskiirte jälitusel põhinev meetod stseeni nende osade
määramiseks, mis tuleb kuvada saadaval pildil.

MÄRKUS: Jälitus võib hõlmata valguse peegeldumist ja murdumist.\cite{ISO:2382-13}
\end{quote}

Kiirtejälitus pakub praeguseni teadaolevatest ilmestusmeetoditest kõige
realistlikumat tulemust, sest kasutab ilmestusel valguse füüsikalist
mudelit.

\section{Kiirtejälituse algoritm}
Esimese kiirtejälituse algoritmi (mitte \textit{ray tracing} aga
\textit{ray casting}) lõi Arthur Appel 1968. aastal. Programm
\ref{prog:raycasting} on selle algoritmi teostus. \texttt{eye} on
kaamera koordinaadid ruumis, \texttt{imagePlane(x, y)} tagastab punkti
pilditasandil koordinaatidega \texttt{x} ja \texttt{y} ja
\texttt{trace(r)} leiab lähima keha selle kiire teel ja tagastab selle
värvi.

\begin{lstlisting}[caption=\textit{Ray casting}, label=prog:raycasting]
int w, h;
Pixel image[h][w];

for(int y = 0; y < h; y++)
{
	for(int x = 0; x < w; x++)
	{
		Ray r;
		r.position = eye;
		r.direction = imagePlane(x, y) - eye;
		image[y][x] = trace(r);
	}
}
\end{lstlisting}

Kasutades kahendpuud kehade seast otsimiseks on algoritmi keerukuseks
\(O(\log n)\) kehade arvu suhtes.

1979. aastal lisas Turner Whitted algoritmile varjud ja valguse
peegeldumise ning murdumise.

1986. aasta SIGGRAPHil tutvustas Jim Kajiya oma artiklit
ilmestusvõrrandist (võrrand \ref{eq:kajiya}), mis üldistas mõndasid
ilmestusmeetodeid. \cite{kajiya86}

\begin{equation} \label{eq:kajiya}
I(x, x') = g(x, x')\left[\epsilon(x, x') +\int_S \rho(x, x', x'')I(x', x'') dx'' \right]
\end{equation}

Kajiya ilmestusvõrrandi üks puuduseid on sõltuvus kehade arvust, sest
integreeritakse üle kõikide kehade kõikide punktide ja liiasus mõningate
\(x''\) korral, näiteks kui leidub mitu \(x''\), mis asuvad sama sirgel
ning lähim neist varjab ära ülejäänud. Tänapäevasemal ilmestusvõrrandil
(võrrand \ref{eq:render}) pole neid puuduseid.

\begin{equation} \label{eq:render}
L(x, \omega) = L_e(x, \omega) + \int_\Omega f_r(x, \omega', \omega) L_i(x, \omega') d\omega'
\end{equation}

\section{Monte Carlo meetod}
\section{Lõikumised}
\section{Valguse peegeldumine}

\chapter*{Kokkuvõte}
\addcontentsline{toc}{chapter}{Kokkuvõte}

% Kasutatud materjalid
\renewcommand\bibname{Kasutatud materjalid}
\bibliographystyle{apacite}
\bibliography{paper}
\end{document}
